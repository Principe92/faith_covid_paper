\section{Introduction}
\label{sec:intro}
The impact of the global pandemic on every area of society is still an ongoing study ranging from family interactions to work culture and faith. In the post-pandemic future of work, nine out of ten organizations will be combining remote and on-site working, according to a new McKinsey survey of 100 executives across industries and geographies~\cite{andrea:alexander}. This marks a major shift in the work culture of most organizations. Before the COVID-19 pandemic, the majority of organizations required employees to spend most of their time on-site. Most places of worship still met in person for their different meetings. With the advent of the internet, faith institutions have gradually been increasing their online presence, to stay up to date with the times. Nevertheless, most of their activities still required physical presence. The use of technology in places of worship is not a new concept. For example, almost every sanctuary in a Canadian Presbyterian church building over 100 years old demonstrates the evolution of technologies related to worship~\cite{260105620200101}. If you walk into, let's say St. Andrew’s Presbyterian Church in Fergus, Ontario, Canada, a careful examination will reveal the different technological additions over the years. The organ, sound speakers, computer, and LCD projectors, to name a few, are all examples of common gadgets found. 

The relationship between religion and technology is as old as technology itself. The influence of one over the other has been an ongoing debate in the scholarly world. This is not a surprise given how both have a tremendous impact on our lives as beings on this planet. The scholarly conversation about this relationship has been ongoing since the mid-twentieth century~\cite{Sacasas}. The nature of this relationship has mostly revolved around the following set of concerns according to Sacasas: religion’s role in determining Western society’s posture toward the natural world, religions’ role in abetting technological development, religion’s role in shaping Western attitudes toward “labor and labor’s tools,” and, more recently, the use of religious language and categories to describe technology ~\cite{Sacasas}. The earliest comments that we have concerning this relationship are from Jacques Ellul in his book, \textit{The Technological Society} published first in French in 1954. The first English translation appeared in 1964~\cite{Sacasas}. In this book, Ellul examines the relationship between Christianity and technology. In his words, the emergence of Christianity marked the "breakdown of Roman technique in every area — on the level of organization as well as in the construction of cities, in industry, and in transport.”~\cite{Sacasas} He also challenged the earlier beliefs that Christianity paved the way for technical development. Of these beliefs, he challenged two in particular: First, that Christianity’s suppression of slavery gave impetus to the development of technology to relieve the miseries of manual labor. Second, Christianity’s disenchantment of the natural world removed metaphysical and psychological obstacles to its technologically enabled exploitation~\cite{Sacasas}. Concerning the first argument, Ellul argued that it fails to account for the impressive technical achievements of slave societies. Regarding the second, his views were that it ignores the other strictures Christian faith placed on technical activity, namely its other-worldly and ascetic tendencies~\cite{Sacasas}. Ellul's comments point to the earliest sentiments on record about the effects religion had on technology and vice-versa. Ellul believes that Christianity, as it was practiced through the late Medieval period, was at best ambivalent to the advance of technology~\cite{Sacasas}.

However, the terms around the current debate were not set by Ellul. That honor goes to an American historian by the name of Lynn White Jnr. In his book, \textit{The Historical Roots of Our Ecological Crisis}, White took an entirely different approach to the narrative established by Ellul regarding the relationship between religion and technology. He started by describing the gap in technical achievement that opened up between Western Europe and both Islamic and Byzantine civilizations to the east. This gap predated the “Scientific Revolution” of the sixteenth century and was already evident by the late Middle Ages. Consequently, White turns to the Middle Ages to understand the nature of Western technology~\cite{Sacasas}. He wanted to understand the cultural influences that conditioned the development and deployment of technology that were adversarial to nature. In his words, the introduction of the heavy plow, which he attributed to being the catalyst for the changing attitudes about humanity’s relationship to nature, “attacked the land with such violence that cross-plowing was not needed.”~\cite{Sacasas} After looking at the creation narrative in the opening chapter of the book of Genesis, White contrasts Christianity to ancient paganism and the Eastern religions. He finds that Christianity “not only established a dualism of man and nature but also insisted that it is God’s will that man exploit nature for his proper ends.”~\cite{Sacasas}. As such, the freedom to exploit nature for his proper ends led to the psychic "technological" revolution, disenchanting nature. Lynn identifies  Christianity as the most important cultural factor driving technological activity in the West and links the historical question to environmental concerns~\cite{Sacasas}. He further developed his thesis in a long 1971 article, \textit{“Cultural Climates and Technological Advance in Middle Ages,”} in which he directly set out to identify the sources of the unprecedented “technological thrust of the medieval West.”~\cite{Sacasas}

So much of the earlier texts by Jacques Ellul and Lynn White Jnr. centered around the influence Christianity, as practiced in the medieval age, had on the technological advancement in the West. Susan White’s study, \textit{Christian Worship and Technological Change}, stands apart as a consideration of technology’s influence on Christian liturgy~\cite{Sacasas}. In her study, she discusses the role of astronomical technology — the astrolabe, the albion, and the rectangulus — in refining the Christian liturgical calendar which had been in acknowledged disarray as well as the role of the mechanical clock in re-calibrating the rhythms of the liturgy~\cite{Sacasas}. In a subsequent chapter, \textit{“Liturgy and Mechanization,”} White notes how transformations in Christian liturgical practice during the nineteenth and twentieth centuries were often inspired by the goals and logic of machine technology. In her analogy, liturgy became a form of mechanistic technology~\cite{Sacasas}.

Fast forward today, the age of the \textit{World Wide Web (aka Internet)}, one would find that as stated by Helland, one of the greatest difficulties in studying religion on the Internet is keeping pace with its rapid developments and changes~\cite{helland}. In this age, there is a distinction between an online religion and religion online. The former allows people to act with unrestricted freedom and high level of interactivity, while the latter only provides religious information with little to no interaction. 

Tafsir (exegesis) is one of the most basic sciences in Islam and it can be divided into three periods, which include the formative, middle, and modern-contemporary periods. During the formative and middle periods, interpretations were compiled manually. The al-Quran was interpreted to rely on the ability of commentators to analyze both language skills, memorization, ratios, and information~\cite{PUTRA2020101418}. These interpreters relied on manual tools like the book of \textit{mufradat al-Fadz al-Quran}, and \textit{fathurrahman}. Today, \textit{tafsir} has developed to the point that it can be studied scientifically via technology. Lajnah Pentashih Mushaf al-Quran (LMPQ), an institution under the umbrella of the Ministry of Religious Affairs in Indonesia, released an exegesis of the al-Quran by using audiovisual technology, applications, and digital data~\cite{PUTRA2020101418}. \textit{Maktabah syamilah} is a digital library that is widely used by Islamic scholars to find literature sources that contain more than 6688 books that are always updated in number~\cite{PUTRA2020101418}. \textit{Lafzi} is a search engine for verses of the Quran based on phonetic similarity~\cite{PUTRA2020101418}.

Despite the predominant use of technology in most places of worship, their online presence, at best, is a working website and some social media presence. Some would say it is due to a lack of will or talent. However, Pillay notes that for a long time many churches have resisted change and spoke against considering or even implementing different ideas of ‘being’ church today. The quest has been to maintain the traditional beliefs and ways of maintaining the practice of churches~\cite{jerry:pillay}.
 





%%% Local Variables:
%%% mode: latex
%%% TeX-master: "main"
%%% End:

%  LocalWords:  biometrics cryptographic parallelized lossy
