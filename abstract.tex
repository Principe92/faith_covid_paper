\begin{abstract}
The impact of the SARS-CoV-2 (aka COVID-19) virus on world society is an ongoing research topic. There is no denying that one vital aspect of its impact is how it put digitization on extra gear. From schools to homes and places of worship, COVID-19 fast-tracked digitization as society came to a halt confining us to our homes for a better part of a year. Online shopping skyrocketed, schools were shut down and forced to ramp up or build their online offerings. Society turned to social media to stay connected and abreast of the news. Working from home suddenly became a requirement as employees went home to curb the pandemic. What about places of worship? Well, they too were forced to go online to host services, stay in touch with members, and offer online payment options to keep collecting dues, gifts, and offerings, a vital component to most faith centers. From churches to mosques, the pandemic forced faith leaders to figure out how to adapt to the sudden change in society. To stay open and relevant, every faith center needed an online presence of some kind. I, personally, spearheaded such sudden digitization at my local church. Within 2 – 3 months, we went from no internet connection to about 70\% coverage, a functioning website with online giving options, and weekly live streaming of our services. With these changes looking to stay with us, I will be exploring how this has changed how we worship across different faiths in society.

\end{abstract}
